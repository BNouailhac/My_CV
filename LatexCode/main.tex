
%-----------------------------------------------------------------------------------------------------------------------------------------------%
%	The MIT License (MIT)
%
%	Copyright (c) 2021 Baptiste Nouailhac
%
%	Permission is hereby granted, free of charge, to any person obtaining a copy
%	of this software and associated documentation files (the "Software"), to deal
%	in the Software without restriction, including without limitation the rights
%	to use, copy, modify, merge, publish, distribute, sublicense, and/or sell
%	copies of the Software, and to permit persons to whom the Software is
%	furnished to do so, subject to the following conditions:
%	
%	THE SOFTWARE IS PROVIDED "AS IS", WITHOUT WARRANTY OF ANY KIND, EXPRESS OR
%	IMPLIED, INCLUDING BUT NOT LIMITED TO THE WARRANTIES OF MERCHANTABILITY,
%	FITNESS FOR A PARTICULAR PURPOSE AND NONINFRINGEMENT. IN NO EVENT SHALL THE
%	AUTHORS OR COPYRIGHT HOLDERS BE LIABLE FOR ANY CLAIM, DAMAGES OR OTHER
%	LIABILITY, WHETHER IN AN ACTION OF CONTRACT, TORT OR OTHERWISE, ARISING FROM,
%	OUT OF OR IN CONNECTION WITH THE SOFTWARE OR THE USE OR OTHER DEALINGS IN
%	THE SOFTWARE.
%	
%
%-----------------------------------------------------------------------------------------------------------------------------------------------%


%============================================================================%
%
%	DOCUMENT DEFINITION
%
%============================================================================%

\documentclass[10pt,A4,english]{article}	


%----------------------------------------------------------------------------------------
%	ENCODING
%----------------------------------------------------------------------------------------

% we use utf8 since we want to build from any machine
\usepackage[utf8]{inputenc}		
\usepackage[USenglish]{isodate}
\usepackage{fancyhdr}
\usepackage[numbers]{natbib}

\usepackage{ragged2e}
\justifying


%----------------------------------------------------------------------------------------
%	LOGIC
%----------------------------------------------------------------------------------------

% provides \isempty test
\usepackage{xstring, xifthen}
\usepackage{enumitem}
\usepackage[english]{babel}
\usepackage{blindtext}
\usepackage{pdfpages}
\usepackage{changepage}
%----------------------------------------------------------------------------------------
%	FONT BASICS
%----------------------------------------------------------------------------------------

% some tex-live fonts - choose your own

%\usepackage[defaultsans]{droidsans}
%\usepackage[default]{comfortaa}
%\usepackage{cmbright}
\usepackage[default]{raleway}
%\usepackage{fetamont}
%\usepackage[default]{gillius}
%\usepackage[light,math]{iwona}
%\usepackage[thin]{roboto}

\usepackage{fontsize}

% set font default
\renewcommand*\familydefault{\sfdefault} 	
\usepackage[T1]{fontenc}

% more font size definitions
\usepackage{moresize}

\usepackage[colorlinks=true, linkcolor=blue, urlcolor=blue]{hyperref}
%----------------------------------------------------------------------------------------
%	FONT AWESOME ICONS
%---------------------------------------------------------------------------------------- 

% include the fontawesome icon set
\usepackage{fontawesome}

% use to vertically center content
% credits to: http://tex.stackexchange.com/questions/7219/how-to-vertically-center-two-images-next-to-each-other
\newcommand{\vcenteredinclude}[1]{\begingroup
\setbox0=\hbox{\includegraphics{#1}}%
\parbox{\wd0}{\box0}\endgroup}
\newcommand{\tab}[1]{\hspace{.2\textwidth}\rlap{#1}}
% use to vertically center content
% credits to: http://tex.stackexchange.com/questions/7219/how-to-vertically-center-two-images-next-to-each-other
\newcommand*{\vcenteredhbox}[1]{\begingroup
\setbox0=\hbox{#1}\parbox{\wd0}{\box0}\endgroup}

% icon shortcut
\newcommand{\icon}[3] { 							
	\makebox(#2, #2){\textcolor{maincol}{\csname fa#1\endcsname}}
}	


% icon with text shortcut
\newcommand{\icontext}[4]{ 						
	\vcenteredhbox{\icon{#1}{#2}{#3}}  \hspace{2pt}  \parbox{0.9\mpwidth}{\textcolor{#4}{#3}}
}

% icon with website url
\newcommand{\iconhref}[5]{ 						
    \vcenteredhbox{\icon{#1}{#2}{#5}}  \hspace{2pt} \href{#4}{\textcolor{#5}{#3}}
}

% icon with email link
\newcommand{\iconemail}[5]{ 						
    \vcenteredhbox{\icon{#1}{#2}{#5}}  \hspace{2pt} \href{mailto:#4}{\textcolor{#5}{#3}}
}

% icon with email link
\newcommand{\iconDate}[5] {
    \vcenteredhbox{\icon{#1}{#2}{#5}}  \hspace{2pt} \href{#4}{\textcolor{#5}{#3}}
}

%----------------------------------------------------------------------------------------
%	PAGE LAYOUT  DEFINITIONS
%----------------------------------------------------------------------------------------

% page outer frames (debug-only)
% \usepackage{showframe}		

% we use paracol to display breakable two columns
\usepackage{paracol}
\usepackage{tikzpagenodes}
\usetikzlibrary{calc}
\usepackage{lmodern}
\usepackage{multicol}
\usepackage{lipsum}
\usepackage{atbegshi}
% define page styles using geometry
\usepackage[a4paper]{geometry}

% remove all possible margins
\geometry{top=1cm, bottom=1cm, left=1cm, right=1cm}

\usepackage{fancyhdr}
\pagestyle{empty}

% space between header and content
% \setlength{\headheight}{0pt}

% indentation is zero
\setlength{\parindent}{0mm}

%----------------------------------------------------------------------------------------
%	TABLE /ARRAY DEFINITIONS
%---------------------------------------------------------------------------------------- 

% extended aligning of tabular cells
\usepackage{array}

% custom column right-align with fixed width
% use like p{size} but via x{size}
\newcolumntype{x}[1]{%
>{\raggedleft\hspace{0pt}}p{#1}}%


%----------------------------------------------------------------------------------------
%	GRAPHICS DEFINITIONS
%---------------------------------------------------------------------------------------- 

%for header image
\usepackage{graphicx}

% use this for floating figures
% \usepackage{wrapfig}
% \usepackage{float}
% \floatstyle{boxed} 
% \restylefloat{figure}

%for drawing graphics		
\usepackage{tikz}			
\usepackage{ragged2e}	
\usetikzlibrary{shapes, backgrounds,mindmap, trees}

%----------------------------------------------------------------------------------------
%	Color DEFINITIONS
%---------------------------------------------------------------------------------------- 
\usepackage{transparent}
\usepackage{color}

% primary color
\definecolor{maincol}{RGB}{ 64,64,64}

% accent color, secondary
% \definecolor{accentcol}{RGB}{ 250, 150, 10 }

% dark color
\definecolor{darkcol}{RGB}{ 70, 70, 70 }

% light color
\definecolor{lightcol}{RGB}{245,245,245}

\definecolor{accentcol}{RGB}{59,77,97}



% Package for links, must be the last package used
\usepackage[hidelinks]{hyperref}

% returns minipage width minus two times \fboxsep
% to keep padding included in width calculations
% can also be used for other boxes / environments
\newcommand{\mpwidth}{\linewidth-\fboxsep-\fboxsep}
	
\usepackage{overpic}    % Pour superposer du texte sur une image
\usepackage{adjustbox} 

%============================================================================%
%
%	CV COMMANDS
%
%============================================================================%

%----------------------------------------------------------------------------------------
%	 CV LIST
%----------------------------------------------------------------------------------------

% renders a standard latex list but abstracts away the environment definition (begin/end)
\newcommand{\cvlist}[1] {
	\begin{itemize}{#1}\end{itemize}
}

%----------------------------------------------------------------------------------------
%	 CV TEXT
%----------------------------------------------------------------------------------------

% base class to wrap any text based stuff here. Renders like a paragraph.
% Allows complex commands to be passed, too.
% param 1: *any
\newcommand{\cvtext}[1] {
	\begin{tabular*}{1\mpwidth}{p{0.98\mpwidth}}
		\parbox{1\mpwidth}{#1}
	\end{tabular*}
}
\newcommand{\cvtextsmall}[1] {
	\begin{tabular*}{0.8\mpwidth}{p{0.8\mpwidth}}
		\parbox{0.8\mpwidth}{#1}
	\end{tabular*}
}
%----------------------------------------------------------------------------------------
%	CV SECTION
%----------------------------------------------------------------------------------------

% Renders a a CV section headline with a nice underline in main color.
% param 1: section title
\newcommand{\cvsection}[1] {
	\vspace{5pt}
	\cvtext{
		\textbf{\Large{\textcolor{darkcol}{#1}}}\\[-4pt]
		\textcolor{accentcol}{ \rule{0.7\textwidth}{1.5pt} } \\
	}
}

\newcommand{\cvsectionsmall}[1] {
	\vspace{5pt}
	\cvtext{
		\textbf{\Large{\textcolor{darkcol}{#1}}}\\[-4pt]
		\textcolor{accentcol}{ \rule{0.3\textwidth}{1.5pt} } \\
	}
}

\newcommand{\cvheadline}[1] {
	\vspace{5pt}
	\cvtext{
		\textbf{\Huge{\textcolor{accentcol}{#1}}}\\[-4pt]
	}
}

\newcommand{\cvsubheadline}[1] {
	\vspace{5pt}
	\cvtext{
		\textbf{\huge{\textcolor{darkcol}{#1}}}\\[-4pt]
		 
	}
}
%----------------------------------------------------------------------------------------
%	META SKILL
%----------------------------------------------------------------------------------------

% Renders a progress-bar to indicate a certain skill in percent.
% param 1: name of the skill / tech / etc.
% param 2: level (for example in years)
% param 3: percent, values range from 0 to 1
\newcommand{\cvskill}[3] {
	\begin{tabular*}{1\mpwidth}{p{0.72\mpwidth}  r}
 		\textcolor{black}{\textbf{#1}} & \textcolor{maincol}{#2}\\
	\end{tabular*}%
	
	\hspace{4pt}
	\begin{tikzpicture}[scale=1,rounded corners=2pt,very thin]
		\fill [lightcol] (0,0) rectangle (1\mpwidth, 0.15);
		\fill [accentcol] (0,0) rectangle (#3\mpwidth, 0.15);
  	\end{tikzpicture}%
}


%----------------------------------------------------------------------------------------
%	 CV EVENT
%----------------------------------------------------------------------------------------

% Renders a table and a paragraph (cvtext) wrapped in a parbox (to ensure minimum content
% is glued together when a pagebreak appears).
% Additional Information can be passed in text or list form (or other environments).
% the work you did
% param 1: time-frame i.e. Sep 14 - Jan 15 etc.
% param 2:	 event name (job position etc.)
% param 3: Customer, Employer, Industry
% param 4: Short description
% param 5: work done (optional)
% param 6: technologies include (optional)
% param 7: achievements (optional)
\newcommand{\cvevent}[7] {
	
	% we wrap this part in a parbox, so title and description are not separated on a pagebreak
	% if you need more control on page breaks, remove the parbox
	\parbox{\mpwidth}{
		\begin{tabular*}{1\mpwidth}{p{0.66\mpwidth}  r}
	 		\textcolor{black}{\textbf{#2}} & \colorbox{accentcol}{\makebox[0.32\mpwidth]{\textcolor{white}{\textbf{#1}}}} \\
			\textcolor{maincol}{#3} & \\
		\end{tabular*}\\[8pt]

        \vspace{-15pt}
	
		\ifthenelse{\isempty{#4}}{}{
			\cvtext{#4}\\
		}
	}
	\vspace{5pt}
}


%----------------------------------------------------------------------------------------
%	 CV META EVENT
%----------------------------------------------------------------------------------------

% Renders a CV event on the sidebar
% param 1: title
% param 2: subtitle (optional)
% param 3: customer, employer, etc,. (optional)
% param 4: info text (optional)
\newcommand{\cvmetaevent}[4] {
	\textcolor{maincol} { \cvtext{\textbf{\begin{flushleft}#1\end{flushleft}}}}

    \vspace{-5pt}

	\ifthenelse{\isempty{#2}}{}{
	\textcolor{black} {\cvtext{\textbf{#2}} }
	}

    \vspace{-10pt}

	\ifthenelse{\isempty{#3}}{}{
		\cvtext{{ \textcolor{maincol} {#3} }}\\
	}

    \vspace{-15pt}

	\cvtext{#4}\\
}


%---------------------------------------------------------------------------------------
%	QR CODE
%----------------------------------------------------------------------------------------

% Renders a qrcode image (centered, relative to the parentwidth)
% param 1: percent width, from 0 to 1
\newcommand{\cvqrcode}[1] {
	\begin{center}
		\includegraphics[width={#1}\mpwidth]{qrcode}
	\end{center}
}


% HEADER AND FOOOTER 
%====================================
\newcommand\Header[1]{%
\begin{tikzpicture}[remember picture,overlay]
\fill[accentcol]
  (current page.north west) -- (current page.north east) --
  ([yshift=50pt]current page.north east|-current page text area.north east) --
  ([yshift=50pt,xshift=-3cm]current page.north|-current page text area.north) --
  ([yshift=10pt,xshift=-5cm]current page.north|-current page text area.north) --
  ([yshift=10pt]current page.north west|-current page text area.north west) -- cycle;
\node[font=\sffamily\bfseries\color{white},anchor=west,
  xshift=0.7cm,yshift=-0.32cm] at (current page.north west)
  {\fontsize{12}{12}\selectfont {#1}};
\end{tikzpicture}%
}

\newcommand\Footer[1]{%
\begin{tikzpicture}[remember picture,overlay]
\fill[lightcol]
  (current page.south east) -- (current page.south west) --
  ([yshift=-80pt]current page.south east|-current page text area.south east) --
  ([yshift=-80pt,xshift=-6cm]current page.south|-current page text area.south) --
  ([xshift=-2.5cm,yshift=-10pt]current page.south|-current page text area.south) --	
  ([yshift=-10pt]current page.south east|-current page text area.south east) -- cycle;
\node[yshift=0.32cm,xshift=9cm] at (current page.south) {\fontsize{10}{10}\selectfont \textbf{\thepage}};
\end{tikzpicture}%
}


%=====================================
%============================================================================%
%
%
%
%	DOCUMENT CONTENT
%
%
%
%============================================================================%
\begin{document}

\columnratio{0.31}
\setlength{\columnsep}{2.2em}
\setlength{\columnseprule}{4pt}
\colseprulecolor{white}


% LEBENSLAUF HIERE
\AtBeginShipoutFirst{\Header{CV}\Footer{1}}
\AtBeginShipout{\AtBeginShipoutAddToBox{\Header{CV}\Footer{2}}}

\newpage

\colseprulecolor{lightcol}
\columnratio{0.31}
\setlength{\columnsep}{2.2em}
\setlength{\columnseprule}{4pt}
\begin{paracol}{2}


\begin{leftcolumn}
%---------------------------------------------------------------------------------------
%	META IMAGE
%----------------------------------------------------------------------------------------

\includegraphics[width=80pt, height=100pt]{resources/photo.png}	%trimming relative to image size

%---------------------------------------------------------------------------------------
%	META SKILLS
%----------------------------------------------------------------------------------------
\vspace{5pt}

	\fcolorbox{white}{white}{\begin{minipage}[c][1.5cm][c]{1\mpwidth}
		\LARGE{\textbf{\textcolor{maincol}{Baptiste Nouailhac}}} \\[2pt]
		\normalsize{ \textcolor{maincol} {Ingénieur Cybersécurité} }
\end{minipage}} \\

\vspace{-8pt}

%\icontext{CaretRight}{12}{AMSN à Monaco}{black}\\[6pt]
\iconemail{Envelope}{16}{baptiste.nouailhac@pm.me}{baptiste.nouailhac@pm.me}{black}\\[2pt]
\vspace{-2pt}
\iconhref{Flag}{16}{monégasque}{}{black}\\[2pt]
%\vspace{-2pt}
\iconhref{Mobile}{16}{06 80 86 73 46}{}{black}\\[2pt]
\vspace{-2pt}
\iconDate{Calendar}{16}{11 mars 2000}{}{black}\\[2pt]
\vspace{-2pt}
%\\iconhref{File}{16}{Thèse professionnelle}{https://www.dropbox.com/scl/fi/grnpmbjjbvhjn29irooos/CEC_MSCyberSCID_thesis_Baptiste_NOUAILHAC.pdf?rlkey=eq2inscsrw4bxfh3a460krj9b&st=az67vyji&dl=0}{black}\\
\iconhref{Code}{16}{Portfolio}{https://bnouailhac.github.io/Personal-Portfolio/}{black}\\[2pt]
\vspace{-2pt}
\iconhref{Linkedin}{16}{baptiste-nouailhac}{https://www.linkedin.com/in/baptiste-nouailhac/}{black}\\[2pt]
\vspace{-2pt}
\iconhref{Github}{16}{BNouailhac}{https://github.com/BNouailhac}{black}\\

\vspace{-2pt}

\cvsectionsmall{Certifications}

\vspace{-15pt}

\textcolor{maincol} {
  \begin{tabular*}{1\mpwidth}{p{0.98\mpwidth}}
    \parbox{1\mpwidth}{\textbf  {\large Certifications Cisco}}
   \end{tabular*}
}

\vspace{5pt}

\ifthenelse{\isempty{#2}}{}{
    \textcolor{black} {
      \begin{tabular*}{1\mpwidth}{p{0.98\mpwidth}}
        \parbox{1\mpwidth}{\textbf{
        \begin{tabular}{ m{1.25cm} m{3cm} }
            \href{https://www.credly.com/badges/b20a3499-1df9-41a1-a831-ab6684cabdb1}{\includegraphics[width=1cm]{resources/CCNA-1.png}} & \href{https://www.credly.com/badges/b20a3499-1df9-41a1-a831-ab6684cabdb1}{CCNA 1} \\  
            \href{https://www.credly.com/badges/6cfdfa6d-2b26-4602-9fa2-d24f30ffd937}{\includegraphics[width=1cm]{resources/CCNA-2.png}} & \href{https://www.credly.com/badges/6cfdfa6d-2b26-4602-9fa2-d24f30ffd937}{CCNA 2} \\  
            \href{https://www.credly.com/badges/f73c5f80-2cd7-4559-9eb6-a1a836d1601e}{\includegraphics[width=1cm]{resources/CCNA-NS.png}} & \href{https://www.credly.com/badges/f73c5f80-2cd7-4559-9eb6-a1a836d1601e}{Network Security} \\  
        \end{tabular}
        }}
       \end{tabular*}
    }
}

\vspace{5pt}

\textcolor{maincol} {
  \begin{tabular*}{1\mpwidth}{p{0.98\mpwidth}}
    \parbox{1\mpwidth}{\textbf  {\large Certifications Croix-Rouge}}
   \end{tabular*}
}

\vspace{-5pt}

\ifthenelse{\isempty{#2}}{}{
    \textcolor{black} {
      \begin{tabular*}{1\mpwidth}{p{0.98\mpwidth}}
        \parbox{1\mpwidth}{\textbf{
        \begin{tabular}{ m{0.5cm} m{3.8cm} }
            \includegraphics[width=1.25cm]{resources/CRM.png} & 
            \begin{itemize}\footnotesize
                \item \href{https://www.croix-rouge.fr/formation/prevention-et-secours-civique-de-niveau-1-psc1}{Prévention et secours civiques de niveau 1}
                \item \href{https://www.croix-rouge.fr/formation-professionnelle/premiers-secours-en-equipe-de-niveau-1-pse1}{Premiers secours en équipe de niveau 1}
                \item Premiers secours psychologiques
            \end{itemize}\\
        \end{tabular}
        }}
       \end{tabular*}
    }
}

\textcolor{maincol} { 
  \begin{tabular*}{1\mpwidth}{p{0.98\mpwidth}}
    \parbox{1\mpwidth}{\textbf  {\large \href{https://www.etsglobal.org/fr/en/digital-score-report/2C4EE79E1F7887F6B4483C94792F6331ACA034CC767A358B70CCBC56D841D5A2MzdmeHRjWk9nMVJWb3JvWFl2cVUxcmVMQk9tUWpHWlVGNlU5WjJEMVhiZEVzVnlh}{TOEIC® Listening and Reading (score 845)} }}
   \end{tabular*}
}

\cvsectionsmall{Langues}

\vspace{-20pt}

\begin{center} % Centre l'ensemble
    \begin{tabular}{m{1.5cm} m{1.5cm} m{1.5cm}} % Crée une grille avec 3 colonnes
        % Image 1
        \begin{overpic}[width=1cm]{resources/france.png}
            \put(-20, 90){\textbf{Maternel}} % Positionne le texte au-dessus
        \end{overpic}
        &
        % Image 2
        \begin{overpic}[width=1cm]{resources/britain.png}
            \put(27.5, 90){\textbf{B2}}
        \end{overpic}
        &
        % Image 3
        \begin{overpic}[width=1cm]{resources/china.png}
            \put(2.5, 90){\textbf{HSK2}}
        \end{overpic}
    \end{tabular}
\end{center}

\vspace{-12pt}

\cvsectionsmall{Extra-professionnel}

\vspace{-25pt}

\begin{itemize}
    \item Secouriste bénévole Croix-Rouge
    \item Veille Cyber
    \item Activité sportive régulière
\end{itemize}

\vspace{-7pt}

\cvsectionsmall{Soft Skills}

\vspace{-25pt}

\begin{itemize}
    \item Travail d’équipe
    \item Gestion de Projet
    \item Autonomie
    \item Adaptation
\end{itemize}

\newpage

%\cvqrcode{0.3}

\end{leftcolumn}
\begin{rightcolumn}
%---------------------------------------------------------------------------------------
%	TITLE  HEADER
%----------------------------------------------------------------------------------------

%---------------------------------------------------------------------------------------
%	WORK EXPERIENCE
%----------------------------------------------------------------------------------------

\cvsection{Expériences professionnelles}

\vspace{-20pt}

\cvevent
{03/2025 - 05/2025}
	{Développeur Front (3 mois)}
	{\href{http://www.myitmanager.org/}{MyItManager}}
	{\begin{itemize}
        \setlength{\parskip}{-1pt}
        \item Développement d'un dashboard pour le Grand Prix de Monaco 2025 de contrôle de flux et d'événements provenant de bornes Wifi et Pads Zebra connectés.
        \item Gestion des données par API pour la visualisation de statistiques détaillées par pad ou par porte, et gestion des incidents (\textbf{C\#/.NET}, \textbf{WPF}, \textbf{Telerik}, \textbf{Javascript})
    \end{itemize}}
	\vfill\null

\vspace{-12pt}

\cvevent
{04/2024 - 10/2024}
	{Ingénieur Cybersécurité (7 mois)}
	{\href{https://amsn.gouv.mc/}{Agence Monégasque de Sécurité Numérique (AMSN)}}
	{\begin{itemize}
        \setlength{\parskip}{-1pt}
        %\item Stage de 7 mois réalisé dans le cadre de mon Mastère Spécialisé à l’agence nationale de cybersécurité de l’Etat de Monaco (équivalent de l’ANSSI)
        \item Développement d'outils de création de règles de détection automatisées pour les sondes de détection (IDS) utilisées par l'Agence (\textbf{Python}, \textbf{MISP}, \textbf{Suricata}, \textbf{VMware}). \href{https://www.dropbox.com/scl/fi/grnpmbjjbvhjn29irooos/CEC_MSCyberSCID_thesis_Baptiste_NOUAILHAC.pdf?rlkey=eq2inscsrw4bxfh3a460krj9b&e=1&st=az67vyji&dl=0}{\textbf{\textit{Sujet de thèse professionnelle}}}
        \item Réalisation de mission d'analyste SOC de niveau 1 en soutient du SOC-MC (\textbf{Splunk}, \textbf{VirusTotal}, \textbf{Shodan})
        %\item \href{https://www.dropbox.com/scl/fi/3i69r87uswokbqv6wdjbt/Lettre-Recommandation-AMSN.pdf?rlkey=qkaty6qs8dkguaaamci4r1l1d&st=qvtggy3i&dl=0}{\textbf{\textit{Lettre de recommandation}}} de la part du directeur de l'agence
    \end{itemize}}
	\vfill\null

\vspace{-12pt}

\cvevent
{08/2022 - 08/2023}
	{Développeur Full-Stack (1 an)}
	{\href{https://livmeds.com/}{Livmed's}}
	{\begin{itemize}
        \setlength{\parskip}{-1pt}
        %\item 6 mois de stage puis 6 mois de CDD au sein de la startup Nicoise
        \item Développement continue en Agile des interfaces web, mobile et backend de la société (\textbf{ReactJS}, \textbf{ReactNative}, \textbf{NodeJS}, \textbf{SQL})
        \item Mise en conformité \textbf{RGPD} des données de l'entreprise
    \end{itemize}}
	\vfill\null

\vspace{-12pt}

\cvevent
{2019 - 2021}
	{Développeur stagiaire (9 mois)}
	{\href{https://www.syselio.com/}{Syselio}}
	{\begin{itemize}
        \setlength{\parskip}{-1pt}
        %\item 2 stages de 5 et 4 mois réalisés entre 2019 et 2021 pour la société de prestation de service informatique
        \item Réalisation de missions de développement en \textbf{javascript} et \textbf{ASP.NET} pour l'ESN monégasque.
    \end{itemize}}
	\vfill\null

 \vspace{-18pt}

%---------------------------------------------------------------------------------------
%	EDUCATION
%----------------------------------------------------------------------------------------

\cvsection{Diplômes}
\vspace{-28pt}

\cvmetaevent
{09/2023 - 10/2024}
{\href{https://www.centrale-mediterranee.fr/fr/formation/formations-expertes/mastere-specialiser-cybersecurite-systemes-complexes-industrie}{Mastère Spécialisé® Cybersécurité systèmes complexes Industrie et Défense (CyberSCID)}}
{Centrale Méditerranée et l'École de l‘air et de l’espace}
{\begin{itemize}
    \setlength{\parskip}{-1.5pt}
    \item Formation labellisée par l’ANSSI (Bac+6) en partenariat avec le Commissariat à l’Énergie Atomique et aux Énergies Alternatives (CEA) et le Commandement de la Cyberdéfense (COMCYBER).
    \item Gouvernance et Cybersécurité \textbf{(Ebios, Cartographie, MCO/MCS, CERT/SOC)}
    \item Réseaux et administration \textbf{(Certifications CCNA, Mise en place d'architecture sécurisée)}
    %\item Introduction à l'automatisation \textbf{(PLC/API, SCADA, HMI, pyramide du CIM)}
\end{itemize}}

\vspace{-17.5pt}

\cvmetaevent
{2018 - 2023}
{\href{https://www.epitech.eu/programme-grande-ecole-informatique/}{Titre d'expert en technologies de l'information}}
{EPITECH - Master (Bac+5, RNCP Niveau 7)}
{\begin{itemize}
    \setlength{\parskip}{-1pt}
    \item Spécialisation : Développement logiciel, gestion de projet, innovation technologique
    \item Méthodologie : Apprentissage par projets, autonomie, adaptabilité, 2 ans d'expérience en entreprise
    %\item Langues: \textbf{C/C++}, \textbf{Javascript}, \textbf{Python}, \textbf{Dart/Flutter}, \textbf{SQL}, \textbf{Git}, \textbf{Bash}
\end{itemize}}

\vspace{-18pt}

\cvmetaevent
{08/2021 - 02/2022}
{\href{http://en.njtu.edu.cn/}{Graduate Student Program}}
{Beijing Jiaotong University}
{\begin{itemize}
    \setlength{\parskip}{-2pt}
    \item Un semestre réalisé à  l’université chinoise de Pékin "Beijing Jiaotong
University" dans le cadre de ma quatrième année de cursus à EPITECH.
\end{itemize}}

\vspace{-15pt}

  \cvsection{Compétences}

 \vspace{-25pt}

 \begin{itemize} 
    \item \textit{Language maîtrisées :} \textbf{C/C++}, \textbf{Javascript}, \textbf{Python}, \textbf{Java}, \textbf{Dart/Flutter}, \textbf{SQL}, \textbf{Bash}
    \item \textit{Outils :} \textbf{Git}, \textbf{Splunk}, \textbf{Pandas}, \textbf{Wireshark}, \textbf{MISP}, \textbf{Suricata}, \textbf{VirtualBox}, \textbf{VMware}, \textbf{Windows}, \textbf{Linux}
    \item \textit{Réseaux :} \textbf{VLAN}, \textbf{Routing}, \textbf{Firewall}, \textbf{IPS/IDS}, \textbf{VPN}, \textbf{IPv4/IPv6}
\end{itemize}

%\begin{center}
%\begin{tabular}{m{4cm} m{4cm}} % Crée une grille avec 3 colonnes
        % Image 1
%        \begin{itemize}
%            \setlength{\parskip}{-1.5pt}
%            \item Travail d'équipe
%            \item Gestion de Projet
%            \item Gestion de crise
%        \end{itemize}
%        &
        % Image 2
%        \begin{itemize}
%            \setlength{\parskip}{-1.5pt}
%            \item Autonomie
%            \item Adaptation
%            \item Écoute
%        \end{itemize}
%    \end{tabular}
%\end{center}

\end{rightcolumn}
\end{paracol}


\end{document}